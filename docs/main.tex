%%%%% PREÁMBULO
\documentclass[12pt]{article}
\usepackage{estilosumu} %% Carga los paquetes y configuración de preambulo.sty

%%%%% DOCUMENTO
\begin{document}

\begin{titlepage}
    \begin{center}
        \MakeUppercase{Curso 25/26} \\
        \MakeUppercase{Programación de Arquitecturas Multinúcleo} \\
        \MakeUppercase{Grado en Ingeniería Informática, 4to. curso} \\
        \MakeUppercase{Facultad de Informática. Universidad de Murcia} \\
        
        \includegraphics[width=0.75\linewidth]{imagenes/hcmut.png}

        \Huge
        \noindent\rule{\textwidth}{0.5pt}
        \vspace{0.3cm}
        \textbf{Trabajo práctico 2}\\
        \LARGE \textbf{Programación con OpenMP}
        \vspace{0.1cm}
        \noindent\rule{\textwidth}{0.5pt}


        \large
        \vspace{0.3cm}

        \textbf{Alejandro Tomás Martínez \href{mailto:alejandro.tomasm@um.es}{(alejandro.tomasm@um.es)}}\\

    \end{center}
\end{titlepage}

\tableofcontents
\newpage

\section{Abstract}

\section{Preámbulo: La relevancia actual de una consola obsoleta}

\section{Introducción a la Gameboy Classic (DMG)}
    \subsection{Contexto histórico}
    \subsection{Características}
    \subsection{Arquitectura}
    \subsection{Particularidades del hardware}
        \subsubsection{Transferencias DMA}
        \subsubsection{Tiles, Sprites y codificación 2BPP}
    \subsection{Entorno de trabajo y limitaciones técnicas encontradas}
        \subsubsection{Set de instrucciones limitado}
        \subsubsection{Límite de recursos}
        \subsubsection{VBlank y la compartición de la PPU}

\section{Videojuego planteado}
    \subsection{Concepto e influencias}
    \subsection{Planteamiento de los requisitos a cumplir}
    \subsection{Extra: Presentación del arte}

\section{Implementación y optimizaciones}
    \subsection{Planteamiento general del objetivo a conseguir}
    \subsection{Técnicas comunes para la reducción de memoria ROM}
        \subsubsection{Metaspriting}
        \subsubsection{Sprite tile optimization}
        \subsubsection{Metatiling}
    \subsection{Técnicas para el procesamiento avanzado}
        \subsubsection{ShadowOAM y DMA}
        \subsubsection{Enteros de punto fijo y Subpixel movement}
        \subsubsection{Tablas de consulta de trigonometría y motor de físicas}
        \subsubsection{Generación procedural mapas}

\section{Conclusiones finales}

\section{Bibliografía}

\end{document}